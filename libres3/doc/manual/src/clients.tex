% !TEX root = manual.tex
\chapter{Client configuration}
\section{Authentication}
S3 clients require that you provide an ``Access key'' and a ``Secret key''.
The ``Access key'' is your \SX username, and your ``Secret key'' is your \SX
token (key).

In case you are accessing your \SX cluster using a username and password,
you need to find out your \SX token by running \verb|sxinit| if you haven't
already done so:
\begin{lstlisting}
$ sxinit sx://username@clustername
$ cat $HOME/.sx/clustername/auth/username
\end{lstlisting}

\section{s3cmd}
\label{sec:s3cmd}

You can use the generated s3cfg config file
\footnote{\path{/etc/libres3/libres3.sample.s3cfg}}
 or configure s3cmd from scratch.
Below we assume that your LibreS3
is running on \verb|libres3.example.com| and it supports TLS\@.
The important s3cmd configuration settings are:

\begin{lstlisting}
^\textbf{use\_https}^ True
^\textbf{host\_base}^ libres3.example.com:8443
^\textbf{host\_bucket}^ %(bucket)s.libres3.example.com:8443
^\textbf{access\_key}^ <your-sx-username>
^\textbf{secret\_key}^ <your-sx-key>
^\textbf{ca\_certs\_file}^ s/etc/ssl/certs/libres3.pem
\end{lstlisting}


If you don't use TLS, please use the port 8008 instead of 8443, and set
\verb|use_https| to \verb|False|. Once you've configured s3cmd check
that it properly connects to LibreS3:

\begin{lstlisting}
$ s3cmd ls --debug 2>&1 | grep host
\end{lstlisting}


Supported s3cmd commands:
\begin{description}
    \item[Bucket] \verb|mb|, \verb|rb|, \verb|ls|, \verb|la|, \verb|du|, \verb|info|,
		\verb|setpolicy|, \verb|delpolicy|,
		\verb|multipart|
    \item[Object] \verb|put|, \verb|get|, \verb|del|, \verb|rm|, \verb|sync|, \verb|info|,
        \verb|cp|, \verb|modify|, \verb|mv|, \verb|abortmp|, \verb|listmp|,
        \verb|sign|, \verb|signurl|
\end{description}

\subsection{Encrypted files}

You can set \verb|gpg_passphrase| in \path{.s3cfg} and use
\verb|s3cmd --encrypt| to upload/download encrypted files
\footnote{Accessing files on \SX volumes that use the aes filter is not supported, because
this would require decrypting the files by LibreS3 server-side, and encrypted volumes in \SX
are meant to be used with client-side encryption}.

\subsection{Caveats}

Certificate verification changed in Python version \verb|2.7.9+|, and you'll need
\verb|s3cmd| version \verb|1.5.1| or newer to match.
The \path{.s3cfg} must contain a \verb|ca_certs_file| entry pointing to the certificate
of the LibreS3 server, otherwise certificate verification (and thus HTTPS
connections) will fail.

Note that wildcard TLS certificates only match one level, hence you should avoid
using bucket names which contain dots.

For example a certificate for \verb|*.s3.example.com| is:
\begin{itemize}
\item valid for \verb|a.s3.example.com| 
\item NOT valid for \verb|a.b.s3.example.com| 
\end{itemize}

\section{Python-boto}
\label{sec:python-boto}

S3 clients using Python boto are configured in \path{~/.boto},
or you can use the generated\footnote{\path{/etc/libres3/libres3.sample.boto}} file.
A typical configuration is:

\begin{lstlisting}
[Credentials]
aws_access_key_id=<your-sx-username>
aws_secret_access_key=<your-sx-key>
s3_host=libres3.example.com
s3_port=8443
[Boto]
is_secure = True
ca_certificates_file=/etc/libres3/libres3.pem
https_validate_certificates=True
\end{lstlisting}

Note that setting \verb|s3_host| will override the hostname you give to
applications on the command-line. If you are using an application that allows
setting the S3 hostname on the command-line, you might want to use that instead.

Note that old version of python-boto require the port to be on \verb|s3_host|
instead of \verb|s3_port|.

\subsection{Caveats}

Certificate verification changed in Python version \verb|2.7.9+|, and you'll need
to add \verb|ca_certificates_file| and \verb|https_validate_certificates=True| entries in the \verb|[Boto]| section of
your \path{.boto} file.
Otherwise python-boto applications will keep trying to reconnect, and eventually
fail with:
\begin{lstlisting}
ssl.SSLError: [SSL: CERTIFICATE_VERIFY_FAILED] certificate verify failed (_ssl.c:581)
\end{lstlisting}

The \verb|libres3_setup| generated python-boto configuration files already have
the necessary entries since LibreS3 version \verb|1.1|.

\section{s3fs-fuse}

S3FS can be used to provide a FUSE-based file system backed by LibreS3.

You must specify the \SX username and access token in \path{~/.passwd-s3fs},
the URL for your LibreS3 server with \verb|-o url|, and the certificate
used by LibreS3 in \verb|CURL_CA_BUNDLE|:
\begin{lstlisting}
$ cat >~/.passwd-s3fs <<EOF
admin:0DPiKuNIrrVmD8IUCuw1hQxNqZfJ0hlBUgyckAolodd4C/4r4ecY3QAA
EOF
$ chmod 0600 ~/.passwd-s3fs
$ mkdir ~/libres3-vol1
$ CURL_CA_BUNDLE=/etc/ssl/certs/libres3.pem s3fs -o url=https://libres3.example.com:8443 vol1 ~/libres3-vol1 -o uid=1000
\end{lstlisting}

If you want to access LibreS3 unencrypted for debugging purposes then:
\begin{lstlisting}
$ s3fs -o url=http://libres3.example.com:8008 vol1 ~/libres3-vol1 -o uid=1000
\end{lstlisting}

\subsection{Caveats}

If s3fs fails to connect to LibreS3 it quits without an error message,
and the next time you access the mountpoint you get an error,
including trying to start s3fs again:
\begin{lstlisting}
$ touch ~/libres3-vol1/x
touch: cannot touch '/home/USER/libres3-vol1/x': Transport endpoint is not connected
$ s3fs ...
s3fs: unable to access MOUNTPOINT /home/USER/libres3-vol1: Transport endpoint is not connected
$ fusermount -u ~/libres3-vol1
$ s3fs ...
\end{lstlisting}

You'll have to manually unmount and it is useful to run \verb|s3fs| in the
foreground to see the actual error message:
\begin{lstlisting}
$ fusermount -u ~/libres3-vol1
$ CURL_CA_BUNDLE=/etc/ssl/certs/libres3.pem s3fs -o url=https://libres3.example.com:8443 vol1 ~/libres3-vol1 -o uid=1000 -f
[...]
\end{lstlisting}

Common errors are using the wrong user/token in \path{~/.passwd-s3fs}, or
s3fs reading the wrong passwd-s3fs (from path{/etc} for example), or that you don't have permissions to access the volume, etc.

\section{DragonDisk}
\label{sec:dragondisk}

Add an account configured like this:
\begin{description}
\item [Provider] Other S3 compatible service
\item [Service endpoint] \verb|libres3.example.com| (\verb|s3_host| from \path{libres3.conf})
\item [Access key] your \SX username
\item [Secret key] your \SX secret token
\item [HTTP port] 8008 (\verb|s3_http_port| from \path{libres3.conf})
\item [HTTPS port] 8443 (\verb|s3_https_port| from \path{libres3.conf})
\end{description}

\section{Other clients}
For information on other clients please refer to our wiki:
\url{http://wiki.skylable.com/wiki/LibreS3_Clients}
%%% Local Variables:
%%% mode: latex
%%% TeX-master: "manual"
%%% End:
