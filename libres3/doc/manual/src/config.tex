% !TEX root = manual.tex
\chapter{Configuration}

\section{Requirements}

LibreS3 by default listens on ports 8008 and 8443, which need to be available on a given
IP address.

LibreS3 connects to the \SX cluster via \verb|HTTP(S)|. You can run LibreS3 and \SX
on the same or different hosts.

LibreS3 doesn't store state on the local filesystem, when it does need to share
data with other LibreS3 instances (e.g.\ multipart uploads) it will use an \SX
volume for that purpose.
You can run any number of LibreS3 instances, to provide load-balancing or
failover.

LibreS3 1.2+ requires at least SX version 2.0 for all the features to work
correctly. \verb|libres3_setup| will check if the SX cluster is running
a supported version.

\subsection{DNS zone entry}
\label{sec:dnszone}

S3 buckets require a wildcard \verb|A| record pointing to the IP address (1.2.3.4 below) of the
host running LibreS3, for example:

\begin{lstlisting}
*.libres3.example.com. A 1.2.3.4
libres3.example.com. A 1.2.3.4
\end{lstlisting}

If you don't have control over the DNS you'll have to modify the \path{/etc/hosts} file
of each client machine and add a line for each bucket you want to access:

\begin{lstlisting}
libres3.example.com 1.2.3.4
bucket1.libres3.example.com 1.2.3.4
bucket2.libres3.example.com 1.2.3.4
\end{lstlisting}

You can also add \verb|CNAME| for virtual hosting buckets:
\begin{lstlisting}
bucket.example.net CNAME        bucket.example.net.libres3.example.com.
\end{lstlisting}

You can of course just add directly \verb|A| or \verb|AAAA| entries that point
directly to the LibreS3 server, it only matters that
clients send the desired bucket name in the \verb|Host:| header.

\section{Setting up a LibreS3 node}

Setting up LibreS3 is as simple as running the interactive tool \verb+libres3_setup+.
If you provide the path to an existing \SX cluster configuration file created
by \verb+sxsetup+, most of the settings will be done automatically.

Please make sure the default volume replica count setting is less or equal
to the number of nodes in the \SX cluster.

In the examples below we assume that you
have an \SX cluster already up and running and you want to deploy
LibreS3 on \verb|libres3.example.com|.

Example setup with sxsetup.conf:

\begin{lstlisting}
# libres3_setup  --sxsetup-conf /etc/sxserver/sxsetup.conf
Successfully loaded SX configuration from '/etc/sxserver/sxsetup.conf'

S3 (DNS) name: libres3.example.com

Locking LibreS3 private settings on SX cluster: sx://admin@192.168.1.192:443/
libres3_setup: main: locking cluster for changes

Generating default SSL certificate and key in /etc/ssl/certs/libres3.pem and /etc/ssl/private/libres3.key
Generating a 2048 bit RSA private key
.....+++
..+++
writing new private key to '/etc/ssl/private/libres3.key'
-----
Uploaded generated SSL certificate and key to SX cluster
libres3_setup: main: unlocking cluster for changes
Settings completed

S3 HTTPS port: 8443

S3 HTTP port: 8008

Default volume size [use K, M, G and T suffixes]: 100G

Default volume replica count: 1

Generating '/etc/libres3/libres3.conf'
File '/etc/libres3/libres3.conf' already exists, overwriting
libres3_setup: main: locking cluster for changes
libres3_setup: main: Updating cluster metadata
libres3_setup: main: unlocking cluster for changes
volume_size = 100G
s3_https_port = 8443
s3_http_port = 8008
s3_host = libres3.example.com
replica_count = 1
allow_volume_create_any_user = true
Updating '/etc/libres3/libres3.sample.s3cfg'
Generating '/etc/libres3/libres3.sample.boto'
Updating '/etc/libres3/libres3-insecure.sample.s3cfg'
Generating '/etc/libres3/libres3-insecure.sample.boto'

Do you want to start LibreS3 now? [Y/n] Y
[....] Restarting LibreS3: libres3No libres3_ocsigen found running; none killed.

Loading configuration from /etc/libres3/libres3.conf
Waiting for server to start (5s) ... OK
\end{lstlisting}


Example without sxsetup.conf:
\footnote{you can use \texttt{sxadm node -{}-info /var/lib/sxserver/storage/} on the \SX node to find out the required information}

\begin{lstlisting}
# libres3_setup
Admin key or path to key-file: 0DPiKuNIrrVmD8IUCuw1hQxNqZfJ0hlBUgyckAolodd4C/4r4ecY3QAA

SX server IP/DNS name: sx.example.com

SX server HTTPS port: 443

Run as user: nobody

Run as group: nogroup

S3 (DNS) name: libres3.example.com

Locking LibreS3 private settings on SX cluster: sx://admin@192.168.1.192:443/
libres3_setup: main: locking cluster for changes

Generating default SSL certificate and key in /etc/ssl/certs/libres3.pem and /etc/ssl/private/libres3.key
Generating a 2048 bit RSA private key
.....+++
..+++
writing new private key to '/etc/ssl/private/libres3.key'
-----
Uploaded generated SSL certificate and key to SX cluster
libres3_setup: main: unlocking cluster for changes
Settings completed

S3 HTTPS port: 8443

S3 HTTP port: 8008

Default volume size [use K, M, G and T suffixes]: 100G

Default volume replica count: 1

Generating '/etc/libres3/libres3.conf'
File '/etc/libres3/libres3.conf' already exists, overwriting
libres3_setup: main: locking cluster for changes
libres3_setup: main: Updating cluster metadata
libres3_setup: main: unlocking cluster for changes
volume_size = 100G
s3_https_port = 8443
s3_http_port = 8008
s3_host = libres3.example.com
replica_count = 1
allow_volume_create_any_user = true
Updating '/etc/libres3/libres3.sample.s3cfg'
Generating '/etc/libres3/libres3.sample.boto'
Updating '/etc/libres3/libres3-insecure.sample.s3cfg'
Generating '/etc/libres3/libres3-insecure.sample.boto'

Do you want to start LibreS3 now? [Y/n] Y
[....] Restarting LibreS3: libres3No libres3_ocsigen found running; none killed.

Loading configuration from /etc/libres3/libres3.conf
Waiting for server to start (5s) ... OK
\end{lstlisting}

You can use \verb|--batch| and provide all settings on the command line, see
\verb|libres3_setup --help| for the full list of options.

To start/stop LibreS3\footnote{LibreS3 and \SX will communicate using TLS by
default. For debugging purposes you can configure \SX with \texttt{sxsetup -{}-no-ssl}}:

\begin{lstlisting}
# libres3 start
Starting LibreS3
LibreS3 started successfully
# libres3 status
--- LibreS3 STATUS ---
LibreS3 is running (PID 28245)

--- LibreS3 INFO ---
SSL private key: /etc/ssl/private/libres3.key
LibreS3 logs: /var/log/libres3/
# libres3 stop
Loading configuration from /etc/libres3/libres3.conf
Sending TERM to PID 28245 ... 
Waiting for PID 28245 ...
\end{lstlisting}


If the server doesn't start, please check the log files for details.

That's it --- your LibreS3 cloud storage is already up and running!
You can now connect to it with your favorite S3 client.

\section{Advanced settings}

\subsection{Internal IP addresses}

If you've configured \SX with public and internal IP addresses LibreS3 can only
access \SX via its public IP address, so make sure it is reachable from LibreS3
and that you set the public IP address of \SX in \verb|sx_host|.

\subsection{libres3.conf}

When you run \verb|libres3_setup| it will generate a file
\path{/etc/libres3/libres3.conf} and store settings in SX cluster metadata.

LibreS3 tries to be compatible with the S3 protocol by default, however
sometimes there are features outside of the S3 protocol that would be useful.
These can be configured in \verb|libres3.conf| and all have the \verb|allow_| prefix:
\begin{description}
\item[allow\_list\_all\_volumes (default=true)] only volumes owned by you are
  listed by default, \seeref{sec:listing-buckets}.
\item[allow\_public\_bucket\_index (default=false)] by default you can't list a
  bucket or subdirectory from a browser, \seeref{sec:directory-indexing}.
\item [allow\_volume\_create\_any\_user (default=true)] by default any authenticated
  user can create a bucket that doesn't exist yet. In \SX only admin users can
  perform such an action, and you can configure LibreS3 to provide that
  behaviour, however some S3 applications may not behave correctly anymore.
\end{description}

There are more settings that tune various aspects of LibreS3, an up-to-date
list\footnote{this list doesn't include settings used for debugging/tuning
  low-level aspects} can always be found in the generated \verb|libres3.conf|: each configuration option has
a comment briefly describing its purpose, and settings that are disabled by
default are commented out.

You can customize \verb|libres3.conf| and restart LibreS3 for the changes to take effect.
However if you are running multiple LibreS3 nodes a more convenient way to
update the settings on the entire cluster is:
\begin{lstlisting}
# libres3_setup --update key=value sx://admin@sx.example.com
\end{lstlisting}
And then run \verb|libres3 reload| on each node.
This will store the LibreS3 settings in SX cluster metadata, which LibreS3 reads
on startup or when it receives a SIGHUP signal.

For example to enable public bucket indexing:
\begin{lstlisting}
# libres3_setup --update allow_public_bucket_index=true sx://admin@sx.example.com
Previous configuration:
	volume_size = 10G
	s3_http_port = 8008
	s3_host = libres3.example.com
	replica_count = 2
	allow_volume_create_any_user = true
	allow_public_bucket_index = false

libres3_setup: main: locking cluster for changes
libres3_setup: main: Updating cluster metadata
libres3_setup: main: unlocking cluster for changes

New configuration:
	volume_size = 10G
	s3_http_port = 8008
	s3_host = libres3.example.com
	replica_count = 2
	allow_volume_create_any_user = true
	allow_public_bucket_index = true
\end{lstlisting}

\subsection{SSL certificates}

When you run \verb|libres3_setup| on the first node a self-signed wildcard SSL certificate is
generated, and uploaded to the SX cluster private settings (accessible only to
admin users).
On the next invocation of \verb|libres3_setup| it will download the SSL
certificate and key from the SX cluster (if available), to ensure that the same
certificate is used on all LibreS3 nodes.

To reset the SSL certificate and allow generating a new one:
\begin{lstlisting}
# sxadm cluster --set-param libres3_private= sx://admin@sx.example.com
\end{lstlisting}

%%% Local Variables:
%%% mode: latex
%%% TeX-master: "manual"
%%% End:
